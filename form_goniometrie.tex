\section{GONIOMETRIE} \label{goniometrie}
\hypertarget{goniometrie}{}

\subsection{Goniometrische getallen van een hoek} \label{goniometrische_getallen}
\hypertarget{goniometrische_getallen}{}

\subsubsection{In een rechthoekige driehoek} \label{rechthoekige_driehoek}
\hypertarget{rechthoekige_driehoek}{}

\begin{center}
\includegraphics{rechtdriehoek.jpg}
\end{center}
\begin{eqnarray*}
\sin\alpha&=&\ds\Frac{A}{C}\\
\cos\alpha&=&\ds\Frac{B}{C}\\
\mbox{tg}\:\alpha&=&\ds\Frac{\sin\alpha}{\cos\alpha}=\ds\Frac{A}{B}\\
\mbox{cotg}\:\alpha&=&\ds\Frac{1}{\mbox{tg}\alpha}=\ds\Frac{B}{A}
\end{eqnarray*}

\subsubsection{De goniometrische cirkel} \label{goniometrische_cirkel}
\hypertarget{goniometrische_cirkel}{}

\begin{center}
\includegraphics{goncirkel.jpg}
\end{center}

\subsubsection{Bijzondere hoeken} \label{bijzondere_hoeken}
\hypertarget{bijzondere_hoeken}{}

% todo: benamingen voor tg, cot, sec en csc zijn hier juist, maar in rest document anders (met mbox...)
\begin{center}
  \begin{tabular}{c|ccccc}
    $\alpha$ & $0=0\degree$ & $\frac{\pi}{6}=30\degree$ & $\frac{pi}{4}=45\degree$ & $\frac{pi}{3}=60\degree$ & $\frac{pi}{2}=90\degree$\\
    \hline
    $\sin\alpha$ & $0$ & $\frac{1}{2}$ & $\frac{\sqrt{2}}{2}$ & $\frac{\sqrt{3}}{2}$ & $1$\\
    $\cos\alpha$ & $1$ & $\frac{\sqrt{3}}{2}$ & $\frac{\sqrt{2}}{2}$ & $\frac{1}{2}$ & $0$\\
    $\tan\alpha$ & $0$ & $\frac{1}{\sqrt{3}}$ & $1$ & $\sqrt{3}$ & $\infty$\\
    $\cot\alpha$ & $\infty$ & $\sqrt{3}$ & $1$ & $\frac{1}{\sqrt{3}}$ & $0$\\
    $\sec\alpha$ & $1$ & $\frac{2}{\sqrt{3}}$ & $\sqrt{2}$ & $2$ & $\infty$\\
    $\csc\alpha$ & $\infty$ & $2$ & $\sqrt{2}$ & $\frac{\sqrt{2}}{3}$ & $1$\\
  \end{tabular}
\end{center}

Om het goniometrische getal van een hoek in het tweede, derde of vierde kwadrant te vinden herleidt je deze hoek eerst naar een hoek van het eerste kwadrant met de formules van verwante hoeken (zie verder).

\subsubsection{Formules} \label{goniometrische_formules}
\hypertarget{goniometrische_formules}{}

\begin{itemize}
\item \textcolor{green}{Grondformule en afgeleide formules}
\begin{eqnarray*}
\ds\Frac{1}{\sin\alpha} & = & \mbox{cosec}\alpha\\
\ds\Frac{1}{\cos\alpha} & = & \mbox{sec}\alpha\\
\cos^2\alpha+\sin^2\alpha&=&1\\
1+\:\mbox{tg}^2\alpha &=&\:\mbox{sec}^2\alpha\\
1+\:\mbox{cotg}^2\alpha&=&\:\mbox{cosec}^2\alpha
\end{eqnarray*}
\item \textcolor{green}{\hypertarget{verwante_hoeken}{Verwante hoeken}}\label{verwante hoeken}
	\begin{itemize}%verwante hoeken
	\item[*] Tegengestelde hoeken ($\alpha$ en $-\alpha$)
	\begin{eqnarray*}
	\sin(-\alpha)& =&-\sin\alpha\\
	\cos(-\alpha)&=&\cos\alpha\\
	\mbox{tg}\:(-\alpha)&=&-\:\mbox{tg}\:\alpha\\
	\mbox{cotg}\:(-\alpha)&=&-\:\mbox{cotg}\:\alpha
	\end{eqnarray*}
	\item[*] Supplementaire hoeken ($\alpha$ en $\pi-\alpha$)
	\begin{eqnarray*}
	\sin(\pi-\alpha)& =&\sin\alpha\\
	\cos(\pi-\alpha)&=&-\cos\alpha\\
	\mbox{tg}\:(\pi-\alpha)&=&-\:\mbox{tg}\:\alpha\\
	\mbox{cotg}\:(\pi-\alpha)&=&-\:\mbox{cotg}\:\alpha
	\end{eqnarray*}
	\item[*] Complementaire hoeken ($\alpha$ en $\ds\Frac{\pi}{2}-\alpha$)
	\begin{eqnarray*}
	\sin(\ds\Frac{\pi}{2}-\alpha)& =&\cos\alpha\\
	\cos(\ds\Frac{\pi}{2}-\alpha)&=&\sin\alpha\\
	\mbox{tg}\:(\ds\Frac{\pi}{2}-\alpha)&=&\:\mbox{cotg}\:\alpha\\
	\mbox{cotg}\:(\ds\Frac{\pi}{2}-\alpha)&=&\:\mbox{tg}\:\alpha
	\end{eqnarray*}
	\item[*] Antisupplementaire hoeken ($\alpha$ en $\pi+\alpha$)
	\begin{eqnarray*}
	\sin(\pi+\alpha)& =&-\sin\alpha\\
	\cos(\pi+\alpha)&=&-\cos\alpha\\
	\mbox{tg}\:(\pi+\alpha)&=&\:\mbox{tg}\:\alpha\\
	\mbox{cotg}\:(\pi+\alpha)&=&\:\mbox{cotg}\:\alpha
	\end{eqnarray*}
	\item[*] Anticomplementaire hoeken ($\alpha$ en $\ds\Frac{\pi}{2}+\alpha$)
	\begin{eqnarray*}
	\sin(\ds\Frac{\pi}{2}+\alpha)& =&\cos\alpha\\
	\cos(\ds\Frac{\pi}{2}+\alpha)&=&-\sin\alpha\\
	\mbox{tg}\:(\ds\Frac{\pi}{2}+\alpha)&=&-\:\mbox{cotg}\:\alpha\\
	\mbox{cotg}\:(\ds\Frac{\pi}{2}+\alpha)&=&-\:\mbox{tg}\:\alpha
	\end{eqnarray*}
	\end{itemize}%verwante hoeken
\item \textcolor{green}{\hypertarget{som-en_verschilformules}{Som- en verschilformules}}\label{som-en_verschilformules}
\begin{eqnarray*}
\cos(\alpha-\beta)&=&\cos\alpha\,\cos\beta+\sin\alpha\,\sin\beta\\
\cos(\alpha+\beta)&=&\cos\alpha\,\cos\beta-\sin\alpha\,\sin\beta\\	
\sin(\alpha+\beta)&=&\sin\alpha\,\cos\beta+\cos\alpha\,\sin\beta\\
\sin(\alpha-\beta)&=&\sin\alpha\,\cos\beta-\cos\alpha\,\sin\beta\\
\mbox{tg}\:(\alpha+\beta)&=&\ds\Frac{\mbox{tg}\:\alpha+\:\mbox{tg}\:\beta}{1-	\:\mbox{tg}\:\alpha\:\mbox{tg}\:\beta}\\
\mbox{tg}\:(\alpha-\beta)&=&\ds\Frac{\mbox{tg}\:\alpha-	\:\mbox{tg}\:\beta}{1+	\:\mbox{tg}\:\alpha\:\mbox{tg}\:\beta}
\end{eqnarray*}
\item \textcolor{green}{\hypertarget{simpson}{Formules van Simpson}}\label{simpson}
\begin{eqnarray*}
\sin\alpha+\sin\beta&=&2\sin\ds\Frac{\alpha+\beta}{2}\,\cos\ds\Frac{\alpha-\beta}{2}\\
\sin\alpha-\sin\beta&=&2\cos\ds\Frac{\alpha+\beta}{2}\,\sin\ds\Frac{\alpha-\beta}{2}\\
\cos\alpha+\cos\beta&=&2\cos\ds\Frac{\alpha+\beta}{2}\,\cos\ds\Frac{\alpha-\beta}{2}\\
\cos\alpha-\cos\beta&=&-2\sin\ds\Frac{\alpha+\beta}{2}\,\sin\ds\Frac{\alpha-\beta}{2}\\
\end{eqnarray*}
\item \textcolor{green}{\hypertarget{dubbele_hoek}{Formules voor de dubbele hoek}}\label{dubbele_hoek}
\begin{eqnarray*}
\sin(2\alpha)&=&2\sin\alpha\,\cos\alpha\\
\cos(2\alpha)&=&\cos^2\alpha - \sin^2\alpha\\
\mbox{tg}(2\alpha)&=&\ds\Frac{2\:\mbox{tg}\:\alpha}{1-\:\mbox{tg}^2\alpha}
\end{eqnarray*}
\item \textcolor{green}{\hypertarget{t-formules}{t-formules}}\label{t-formules}\newline
Stel $\mbox{tg}\Frac{\alpha}{2}=t$, dan kunnen we $\sin\alpha, \cos\alpha$ en 	$\mbox{tg}\,\alpha$ schrijven in functie van $t$.
\begin{eqnarray*}
\sin\alpha&=&\Frac{2t}{1+t^2}\\
\cos\alpha&=&\Frac{1-t^2}{1+t^2}\\
\mbox{tg}\,\alpha&=&\Frac{2t}{1-t^2}
\end{eqnarray*}
\end{itemize}

\subsubsection{Oplossen van driehoeken} \label{oplossen_driehoeken}
\hypertarget{oplossen_driehoeken}{}
	\begin{itemize}%oplossen van drhn
	\item \textcolor{green}{Rechthoekige driehoeken}
		\begin{itemize}%rechthoekige drhn
		\item[*]De stelling van Pythagoras:
		\[C^2=A^2+B^2\]
		\end{itemize}%rechthoekige drhn
	\item \textcolor{green}{Willekeurige driehoeken}\newline
		%\docLink[tekening]{driehoek.jpg}{\includegraphics{tekening.gif}}
                \includegraphics{driehoek.jpg}
		\begin{itemize}%willekeurige drhn
		\item[*] De \hypertarget{sinusregel}{{\bf sinusregel}}:\label{sinusregel}
		\[\ds\Frac{A}{\sin\alpha}=\ds\Frac{B}{\sin\beta}=\ds\Frac{C}{\sin\gamma}=2R\]
		met $R$ de straal van de omgeschreven cirkel.\newline
		\item[*] De \hypertarget{cosinusregel}{{\bf cosinusregel}}:\label{cosinusregel}
		\begin{eqnarray*}
		A^2&=&B^2+C^2-2BC\cos\alpha\\
		B^2&=&A^2+C^2-2AC\cos\beta\\
		C^2&=&A^2+B^2-2AB\cos\gamma
		\end{eqnarray*}
		\end{itemize}%willekeurige drhn
	\end{itemize}%oplossen van drhn


\subsection{Goniometrische functies}

Zie Sectie~\ref{goniometrische_functies} op pagina~\pageref{goniometrische_functies}.


% Dit werk is gelicenseerd onder een Creative Commons
% Naamsvermelding-GelijkDelen 3.0 Unported.
% Bezoek http://creativecommons.org/licenses/by-sa/3.0/ om een kopie te zien 
% van de licentie of stuur een brief naar Creative Commons, 444 Castro Street, 
% Suite 900, Mountain View, California, 94041, USA.
