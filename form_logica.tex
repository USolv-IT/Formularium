\section{LOGICA} \label{logica}
\hypertarget{logica}{}

\subsection{Verklaring van de gebruikte symbolen} \label{verklaring_symbolen}
\hypertarget{verklaring_symbolen}{}
In wat volgt worden volgende symbolen gebruikt:\vskip 0.5cm
\begin{tabular}{l|l}
symbool & verklaring \\
\hline
$P, Q, R$ & uitspraken \\
$\neg$ & niet \\
$\vee$ & of \\
$\wedge$ & en \\
$\Rightarrow$ & als...dan \\
$\Leftrightarrow$ & als en slechts als \\
$\forall$ & voor alle\\
$\exists$ & er bestaat\\
\end{tabular}\vskip 0.5cm

\subsection{Logische stellingen} \label{logische_stellingen}
\hypertarget{logische_stellingen}{}
\begin{enumerate}%logische stellingen
\item $\neg(\neg P)\:\Leftrightarrow\: P$
\item $(P\:\Rightarrow\: Q) \:\Leftrightarrow\: (\neg P \vee Q)$
\item $((P \:\Rightarrow\: Q)\wedge P)\:\Rightarrow\: Q$
\item $((P \:\Rightarrow\: Q)\wedge \neg Q)\:\Rightarrow\: \neg P$
\item $(\neg(P \:\wedge\: Q)\wedge P)\:\Rightarrow\: \neg Q$
\item $(P \vee Q)\wedge\neg Q)\:\Rightarrow\:P$
\item \textcolor{green}{Contrapositie van de implicatie:}$(P\:\Rightarrow\: Q) \:\Leftrightarrow\: (\neg Q \:\Rightarrow\: \neg P)$
\item \textcolor{green}{\hypertarget{de_morgan}{Wetten van De Morgan:}}\label{de_morgan}\newline
\begin{tabular}{c}
$\neg(P\wedge Q)\:\Leftrightarrow\:\neg P \vee \neg Q $\\
$\neg(P \vee Q)\:\Leftrightarrow\:\neg P \wedge \neg Q$
\end{tabular}
\item \textcolor{green}{Commutativiteiten:}\newline
\begin{tabular}{rcl}
$P \wedge Q$ & $\Leftrightarrow $ & $Q \wedge P$\\
$P \vee Q$ & $\Leftrightarrow $ & $Q \vee P$\\
$(P \:\Leftrightarrow\: Q)$ & $\Leftrightarrow $ & $(Q \:\Leftrightarrow\: P$)\\
\end{tabular}
\item \textcolor{green}{Associativiteiten:}\newline
\begin{tabular}{rcl}
$(P \wedge Q)\wedge R$ & $\Leftrightarrow $ & $P \wedge (Q \wedge R)$\\
$(P \vee Q)\vee R$ & $\Leftrightarrow $ & $P \vee (Q \vee R)$\\
$((P \:\Leftrightarrow\: Q)\:\Leftrightarrow\: R)$ & $\Leftrightarrow $ & $(P\:\Leftrightarrow\: (Q \:\Leftrightarrow\: R))$\\
\end{tabular}
\item \textcolor{green}{Distributiviteiten:}\newline
\begin{tabular}{rcl}
$P \wedge (Q\vee R)$ & $\Leftrightarrow $ & $(P \wedge Q) \vee (P\wedge R)$\\
$P \vee (Q\wedge R)$ & $\Leftrightarrow $ & $(P \vee Q) \wedge (P\vee R)$\\
\end{tabular}
\item \textcolor{green}{Transitiviteiten:}\newline
\begin{tabular}{rcl}
$((P \:\Rightarrow\: Q)\wedge (Q\:\Rightarrow\: R))$ & $ \Rightarrow $ & $(P \:\Rightarrow\: R)$ \\
$((P \:\Leftrightarrow\: Q)\wedge (Q\:\Leftrightarrow\: R))$ & $ \Rightarrow $ & $(P \:\Leftrightarrow\: R)$ \\
\end{tabular}
\end{enumerate}%logische stellingen
\vskip 0.5cm

\subsection{Uitspraakvormen met kwantoren} \label{uitspraakvormen_kwantoren}
\hypertarget{uitspraakvormen_kwantoren}{}
Stel $P(x)$ een uitspraakvorm in de veranderlijke $x$ en $A$ een referentieverzameling.\newline Dan gelden volgende wetten:\vskip 0.5cm
\begin{tabular}{rcl}
$\neg(\forall x \in A:P(x)) $ & $\Leftrightarrow$ & $\exists\: x \in A: \neg P(x)$\\
$\neg(\exists x \in A:P(x)) $ & $\Leftrightarrow$ & $\forall\: x \in A: \neg P(x)$\\
\end{tabular}


% Dit werk is gelicenseerd onder een Creative Commons
% Naamsvermelding-GelijkDelen 3.0 Unported.
% Bezoek http://creativecommons.org/licenses/by-sa/3.0/ om een kopie te zien 
% van de licentie of stuur een brief naar Creative Commons, 444 Castro Street, 
% Suite 900, Mountain View, California, 94041, USA.
