\hypertarget{combinatoriek}{}
\section{COMBINATORIEK} \label{combinatoriek}

\hypertarget{telproblemen}{}
\subsection{Telproblemen} \label{telproblemen}

\hypertarget{variaties}{}
\subsubsection{Variaties} \label{variaties}
\begin{itemize}
\item \textcolor{green}{Wat?}\newline
Een variatie van $p$ elementen uit $n$ elementen $(p\leq n )$ is een {\bf geordend} $p$-tal van $p$ {\bf verschillende } elementen gekozen uit een gegeven verzameling van $n$ elementen.
\item \textcolor{green}{Voorbeeld.} \newline
Enkele variaties van 3 elementen uit$\{a, b, c, d\}$ zijn $abc, abd, acb, acd, \ldots$
\item \textcolor{green}{Formule.} \newline
Het aantal variaties van $p$ elementen uit $n$ elementen $(1\leq p\leq n)$:
\[V^p_n = \displaystyle \Frac{n!}{(n-p)!}\]
\end{itemize}

\hypertarget{herhalingsvariaties}{}
\subsubsection{Herhalingsvariaties} \label{herhalingsvariaties}
\begin{itemize}
\item \textcolor{green}{Wat?} \newline
Een herhalingsvariatie van $p$ elementen uit $n$ elementen is een {\bf geordend} $p$-tal van elementen gekozen uit een gegeven verzameling van $n$ elementen. Eenzelfde element mag meer dan eens voorkomen!
\item \textcolor{green}{Voorbeeld.} \newline
Enkele herhalingsvariaties van 3 elementen uit $\{a, b, c, d\}$ zijn $aaa, aab, abc, aba, acc, \ldots$
\item \textcolor{green}{Formule.} \newline
Het aantal herhalingsvariaties van $p$ elementen uit $n$ elementen is:
\[\bar{V}^p_n=n^p\]
\end{itemize}

\hypertarget{permutaties}{}
\subsubsection{Permutaties} \label{permutaties}
\begin{itemize}
\item \textcolor{green}{Wat?} \newline
Een permutatie van $n$ verschillende elementen is een variatie van $n$ elementen uit $n$ elementen.
\item \textcolor{green}{Voorbeeld.} \newline
Alle permutaties van 3 elementen uit $\{a, b, c\}$ zijn $abc, acb, bac, bca, cab, cba$.
\item \textcolor{green}{Formule.} \newline
Het aantal permutaties van $n$ elementen is:
\[P_n=n!\]
\end{itemize}

\hypertarget{herhalingspermutaties}{}
\subsubsection{Herhalingspermutaties} \label{herhalingspermutaties}
\begin{itemize}
\item \textcolor{green}{Wat?} \newline
Herhalingspermutaties zijn permutaties van $n$ elementen waarbij onder de $n$ elementen dezelfde elementen meerdere malen mogen voorkomen.
\item \textcolor{green}{Voorbeeld.} \newline
Enkele herhalingspermutaties van ${a, a, b, c}$ zijn $aabc, abca, abac,\ldots$.
\item \textcolor{green}{Formule.} \newline
Stel $l_i$ het aantal keer dat elk van de $p$ verschillende elementen $a_i$ voorkomt en $n$ het totaal aantal elementen, dan is het aantal herhalingspermutaties van die $n$ elementen:
\[\bar{P}^{l_1,l_2,\ldots,l_p}_n=\displaystyle\Frac{n!}{l_1!l_2!\ldots l_p!}\]
\end{itemize}

\hypertarget{combinaties}{}
\subsubsection{Combinaties} \label{combinaties}
\begin{itemize}
\item \textcolor{green}{Wat?} \newline
Een combinatie van $p$ elementen uit $n$ elementen $(p \leq n)$ is een deelverzameling van $p$ elementen gekozen uit een gegeven verzameling van $n$ elementen. De volgorde is niet van belang!
\item \textcolor{green}{Voorbeeld.} \newline
Alle combinaties van 3 elementen uit $\{a, b, c, d\}$ zijn: $\{a, b, c\}, \{a, b, d\}, \{a, c, d\}, \{b, c, d\}$
\item \textcolor{green}{Formule.} \newline
Het aantal combinaties van $p$ elementen uit $n$ elementen is:
\[C^p_n=\displaystyle\Frac{n!}{p!(n-p)!}\]
\end{itemize}

\hypertarget{herhalingscombinaties}{}
\subsubsection{Herhalingscombinaties} \label{herhalingscombinaties}
\begin{itemize}
\item \textcolor{green}{Wat?}\newline
Herhalingscombinaties zijn combinaties van $p$ elementen uit $n$ elementen waarbij onder de $n$ elementen dezelfde elementen meerdere malen mogen voorkomen.
\item \textcolor{green}{Voorbeeld.} \newline
Enkele herhalingscombinaties van 7 elementen uit $\{a, a, a, b, b, b, c, c, c\}$ zijn $\{a, a, a, b, b, b, c\}, \{a, a, b, b, b, c, c\}, \ldots$
\item \textcolor{green}{Formule.} \newline
Het aantal herhalingscombinaties van $p$ elementen uit $n$ elementen is:
\[\bar{C}^p_n=C^p_{n+p-1}\]
\end{itemize}

\hypertarget{aantal_deelverzamelingen}{}
\subsubsection{Aantal deelverzamelingen van een verzameling} \label{aantal_deelverzamelingen}

Het aantal deelverzamelingen van een verzameling met $n$ elementen is $2^n$. 

\hypertarget{duivenhokprincipe}{}
\subsubsection{Het duivenhokprincipe} \label{duivenhokprincipe}

Worden er $n$ voorwerpen geplaatst in $r$ laden, met $n>r$, dan is er minstens \'e\'en lade die minstens twee voorwerpen bevat.

\hypertarget{binomium}{}
\subsubsection{Het binomium van Newton} \label{binomium}

In onderstaande formules wordt volgende notatie gebruikt:
\[\left( \begin{array}{c}n\\p \end{array} \right)=\displaystyle \Frac{n!}{p!(n-p)!}\]
Dit noemen we de binomiaalco\"effici\"enten.\newline
\textcolor{green}{De binomiaalformule:}
\[(a+b)^n=\left( \begin{array}{c}n\\0 \end{array} \right)a^n + \left( \begin{array}{c}n\\1 \end{array} \right)a^{n-1}b + \left( \begin{array}{c}n\\2 \end{array} \right)a^{n-2}b^2+\ldots + \left( \begin{array}{c}n\\n-1 \end{array} \right)ab^{n-1}+\left( \begin{array}{c}n\\n \end{array} \right)b^n \]
Of ook nog:
\[(a+b)^n=\sum_{i=0}^n \left( \begin{array}{c}n\\i \end{array} \right)a^{n-i}b^i\]


\hypertarget{kansrekenen}{}
\subsection{Kansrekening} \label{kansrekenen}

\hypertarget{laplace}{}
\subsubsection{Formule van Laplace} \label{laplace}

Voor volgende formules is het belangrijk de begrippen {\it uitkomstenverzameling} en {\it gebeurtenis} te begrijpen.
\begin{itemize}
\item Een \textcolor{green}{uitkomstenverzameling (universum)} is de verzameling van alle mogelijke uitkomsten bij een kansexperiment, bv. bij een dobbelsteen : $U =\{1, 2, 3, 4, 5, 6\}$.
\item Een \textcolor{green}{gebeurtenis} is een deelverzameling van de uitkomstenverzameling, bij het voorbeeld van de dobbelsteen zijn $\{6\}, \{1, 2, 4\}$ mogelijke gebeurtenissen.
\end{itemize}
\textcolor{green}{De formule van Laplace:}\newline
Stel $n$ het aantal elementen van het universum $U$ en $p$ het aantal elementen van de gebeurtenis $A$, dan is de kans voor de gebeurtenis ($P(A)$):\newline\newline
\[P(A)=\displaystyle \Frac{p}{n}\]\newline
Zo is bv. de kans om met een dobbelsteen 3 of 4 te gooien : $P(\{3,4\})= \displaystyle{ \Frac{2}{6}=\Frac{1}{3}}.$\newline\newline\newline

\hypertarget{kanswetten}{}
\subsubsection{Belangrijke kanswetten} \label{kanswetten}

\begin{itemize}
\item Zekere gebeurtenis: $P(U)=1$
\item Onmogelijke gebeurtenis: $P(\emptyset)=0$
\item Zij $\bar A$ (=$U\backslash A$)  het complement van de gebeurtenis $A$ , dan geldt: \[P(A)+P(\bar A)=1\]
\item Zij $A$ en $B$ twee gebeurtenissen, dan geldt: \[P(A\cup B)=P(A) +P(B)-P(A\cap B)\]
\item Gevolg: als $A$ en $B$ disjuncte gebeurtenissen zijn $(A\cap B=\emptyset)$, dan geldt: \[P(A\cup B)=P(A) + P(B)\]
\item Samengestelde experimenten:
\begin{description}
\item [(i)] Afhankelijke deelexperimenten / voorwaardelijke kans:\vskip 0.5cm
\textcolor{green}{Voorbeelden:} Je trekt, zonder teruglegging van de kaarten, twee kaarten uit een spel. Hoe groot is de kans dat je als eerste kaart een harten trekt en als tweede kaart een klaveren?\vskip 0.5cm
\textcolor{green}{Formule:} Stel $p(A\:|\:B)$ de voorwaardelijke kans van A als B reeds gerealiseerd is, dan geldt: \[P(A\cap B)= P(B)\cdot P(A\:|\:B)\]
Uitbreiding: \[P(A\cap B\cap C)=P(A)\cdot P(B\:|\:A)\cdot P(C\:|\:A\cap B)\]
Het voorbeeld wordt dus: $P=\displaystyle\Frac{13}{52}\cdot \displaystyle\Frac{13}{51}$
\item[(ii)] Onafhankelijke deelexperimenten:\vskip 0.5cm
\textcolor{green}{Voorbeelden:} Je trekt, met teruglegging van de kaarten, twee kaarten uit een spel. Wat is de kans dat de eerste kaart een harten is en de tweede een klaveren?
\vskip 0.5cm
\textcolor{green}{Formule:} Zij $A$ en $B$ onafhankelijke gebeurtenissen, dan geldt:
\[P(A\cap B)=P(A)\cdot P(B)\] 
Het voorbeeld wordt dus: $P=\displaystyle\Frac{13}{52}\cdot \displaystyle\Frac{13}{52}$
\end{description}
\end{itemize}


% Dit werk is gelicenseerd onder een Creative Commons
% Naamsvermelding-GelijkDelen 3.0 Unported.
% Bezoek http://creativecommons.org/licenses/by-sa/3.0/ om een kopie te zien 
% van de licentie of stuur een brief naar Creative Commons, 444 Castro Street, 
% Suite 900, Mountain View, California, 94041, USA.
